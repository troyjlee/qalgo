\documentclass[12pt]{article}

\usepackage{times}
\usepackage{fullpage}
\usepackage{amsfonts,amssymb,amsmath,amsthm}
\usepackage{latexsym}
\usepackage{gauss}
\usepackage{tikz}
\usepackage{tikz-cd}
\usepackage{tkz-graph}
\usepackage{calc}
\usepackage{url}
\usepackage{thmtools}
\usepackage{thm-restate}
\usepackage{hyperref}
\usepackage[capitalise, nameinlink]{cleveref}
\usepackage{algorithm}
\usepackage{algorithmicx}
\usepackage{algpseudocode}
%\usepackage[pdftex]{color}
%\usepackage[breaklinks]{hyperref}
\usepackage{graphicx}
\usepackage{enumitem}
\setcounter{MaxMatrixCols}{15}

%%% Basic notation
\newcommand{\N}{\mathbb{N}}
\newcommand{\Q}{\mathbb{Q}}
\newcommand{\R}{\mathbb{R}}
\newcommand{\Z}{\mathbb{Z}}
\newcommand{\F}{\mathbb{F}}
\newcommand{\E}{\mathbb{E}}
\newcommand{\C}{\mathbb{C}}
\newcommand{\zero}{\mathbf{0}}
\newcommand{\Id}{\mathbf{I}}
\newcommand{\vecu}{\vec{u}}
\newcommand{\Acal}{\mathcal{A}}
\newcommand{\Ccal}{\mathcal{C}}
\newcommand{\Hcal}{\mathcal{H}}
\newcommand{\Lcal}{\mathcal{L}}
\newcommand{\Pcal}{\mathcal{P}}
\newcommand{\Rcal}{\mathcal{R}}
\newcommand{\Kcal}{\mathcal{K}}
\newcommand{\M}{\mathcal{M}}
\newcommand{\Scal}{\mathcal{S}}
\newcommand{\Tcal}{\mathcal{T}}
\newcommand{\Fcal}{\mathcal{F}}
\newcommand{\Gcal}{\mathcal{G}}
\newcommand{\Tr}{\mathrm{Tr}}
\newcommand{\bin}{\mathrm{bin}}
\newcommand{\iu}{\mathrm{i}}
\newcommand{\OR}{\mathrm{i}}
\newcommand{\MAJ}{\mathrm{MAJ}}
\newcommand{\symvec}{\mathrm{symvec}}
\newcommand{\symmat}{\mathrm{Sym}}
\newcommand{\Recover}{\mathsf{RecoverOneFromAll}}
\newcommand{\NULL}{\mathrm{NULL}}
\newcommand{\MINCUT}{\mathrm{MINCUT}}
\newcommand{\CON}{\mathrm{CONNECTIVITY}}
\newcommand{\THRESH}{\mathrm{THRESHOLD}}
\newcommand{\Contract}{\mathrm{Contract}}
\newcommand{\ARGMINCUT}{\mathrm{ARGMINCUT}}
\newcommand{\lin}{\mathrm{lin}}
\newcommand{\cut}{\mathrm{cut}}
\newcommand{\ind}{\mathrm{ind}}
\newcommand{\diag}{\mathrm{diag}}
\newcommand{\IP}{\mathrm{IP}}
\newcommand{\atoms}{\mathrm{atoms}}
\newcommand{\PARITY}{\mathrm{PARITY}}
\newcommand{\upto}{\mathbin{:}}
\DeclareMathOperator*{\argmin}{argmin}
\newcommand{\polylog}{\mathrm{polylog}}
\newcommand{\zeros}{\mathrm{zeros}}
\newcommand{\ones}{\mathrm{ones}}
\newcommand{\rmvec}{\mathrm{vec}}
\newcommand{\rmmat}{\mathrm{mat}}
\newcommand{\laspan}{\mathrm{span}}
\newcommand{\cross}{\mathrm{overlap}}
\newcommand{\coisa}{weighted biadjacency matrix }
\newcommand{\card}[1]{|#1|}
\newcommand{\len}[1]{l(#1)}
\newcommand{\floor}[1]{\lfloor #1 \rfloor}
\newcommand{\ceil}[1]{\left\lceil #1 \right\rceil}
\newcommand{\pair}[2]{\langle #1,#2 \rangle}
\newcommand{\triple}[3]{\langle #1,#2,#3 \rangle}
\def\01{\{0,1\}}

\newcommand{\ith}{i^{\scriptsize \mbox{{\rm th}}}}
\newcommand{\jth}{j^{\scriptsize \mbox{{\rm th}}}}

%\newcommand{\dim}{\mathrm{dim}}
\newcommand{\cd}{\mathrm{cdim}}
\newcommand{\sep}{\mathrm{sep}}
\newcommand{\mer}{\mathrm{mer}}

\DeclareMathOperator*{\argmax}{arg\,max}

\newcommand{\norm}[1]{\|#1\|}


\newcommand{\troy}[1]{\textcolor{red}{#1}}

\newcommand{\bra}[1]{\langle#1|}
\newcommand{\ket}[1]{|#1\rangle}
\newcommand{\bigket}[1]{\big|#1\big\rangle}
\newcommand{\braket}[2]{\langle#1, #2\rangle}
\newcommand{\rk}{\mathrm{rk}}
\newcommand{\I}{\mathcal{I}}
\newcommand{\1}{\mathbf{1}}


\newtheorem{theorem}{Theorem}
\newtheorem{question}[theorem]{Question}
\newtheorem{lemma}[theorem]{Lemma}
\newtheorem{corollary}[theorem]{Corollary}
\newtheorem{proposition}[theorem]{Proposition}
\newtheorem{remark}[theorem]{Remark}
\newtheorem{example}[theorem]{Example}
\newtheorem{claim}[theorem]{Claim}
\newtheorem{fact}[theorem]{Fact}
\newtheorem{hypothesis}[theorem]{Hypothesis}

\theoremstyle{definition}
\newtheorem{exercise}[theorem]{Exercise}
\newtheorem{problem}[theorem]{Problem}
\newtheorem{definition}[theorem]{Definition}

\begin{document}
\title{Problem Set 2}
\date{}
\maketitle

\paragraph*{1. Generalized Bernstein-Vazirani}
Let $M,n$ be a positive integers.  Let $s \in \Z_M^n$ and define the 
function $f_s : \Z_M^n \rightarrow \Z_M$ by $f_s(x) = \langle s, x \rangle \mod M$.  
Given access to an oracle $O_{f_s}$ which for $x \in \Z_M^n , b \in \Z_M$ acts as 
$O_{f_s} \ket{x} \ket{b} = \ket{x} \ket{b + f_s(x) \bmod M}$, design a 
quantum algorithm that computes $s$ with one application of $O_{f_s}$.  

Hint: You may want to generalize the ``phase-kickback trick'' to show with the oracle 
$O_{f_s}$ you can also implement an oracle $O'_{f_s}$ with the behavior
\[
O'_{f_s} \ket{x}\ket{b} = \omega^{-f_s(x) \cdot b} \ket{x}\ket{b} 
\]
where $\omega = e^{2\pi \iu/M}$.  

Bonus: What kind of errors in the oracle can your algorithm tolerate (analogous to what we saw in problem 7 of 
problem set 1)?

\paragraph*{2. Continued fractions}
In the classical post-processing of Shor's period finding algorithm we have a fraction $b/N$ and want to find the 
best rational approximation to this number whose denominator is at most $M$.  In lecture we said this can be 
done in polynomial time as the task can be written as a two-variable integer linear program.  Now we see a direct 
way to do this via continued fraction expansion.  A nice discussion of continued fractions, including all the material 
below, can be found in Chapter 10 of Hardy and Wright's An introduction to the theory of numbers.

A finite continued fraction is an expression of the form
\[
a_0 + \frac{1}{a_1 + \frac{1}{a_2 + \frac{1}{a_3 + \frac{1}{\cdots + \frac{1}{a_t}}}}} \enspace.
\]
We will denote this number by $[a_0, \ldots, a_t]$.  For $0 \le j \le t$ we call $[a_0, \ldots, a_j]$ the 
$\jth$ convergent to $[a_0, \ldots, a_p]$.  A continued fraction $[a_0, \ldots, a_t]$ is called \emph{simple}
if $a_1, \ldots, a_p$ are all positive integers ($a_0$ can be non-positive).  Every rational number can be represented by a finite simple 
continued fraction.

Here is an algorithm to find such a representation.  Let $x$ be a positive rational number.  Then set 
\begin{align*}
a_0 = \floor{x}, &\quad x_1 = \frac{1}{x-a_0} \\
a_1 = \floor{x_1}, &\quad x_2 = \frac{1}{x-a_1} \\
a_2 = \floor{x_2}, &\quad x_3 = \frac{1}{x-a_2} \\
&\cdots
\end{align*}
The essential principle at work here is that $x = a_0 + \frac{1}{a_1'}$ where $a_1' = \frac{1}{x-a_0}$.  Then 
since $[a_0, [a_1, \ldots, a_t]] = [a_0, a_1, \ldots, a_t]$ our task becomes to find a continued fraction expansion 
of $a_1'$ which we do by the same procedure.  

One can also find an inductive expression for $[a_0, \ldots, a_j]$.  If 
\begin{align*}
p_0 &= a_0, & p_1 &= a_1 a_0 + 1, & p_j &= a_j p_{j-1} + p_{j-2} \\
q_0 &= 1,  & q_1 &= a_1, & q_j &= a_j q_{j-1} + q_{j-2}
\end{align*}  
then $[a_0, \ldots, a_j] = \frac{p_j}{q_j}$ and this is in lowest terms.  Note that $q_j \ge 2 q_{j-2}$ thus $q_j$ increases 
at least exponentially.  An important property of the continued fraction expansion for the application in Shor's algorithm 
is that if 
\[
|x - \frac{c}{d}| \le |x - \frac{p_j}{q_j}|
\]
then $d \ge q_j$.  

Now the questions:
\begin{enumerate}
\item Find the continued fraction expansion of $\frac{527}{1024}$.  
\item Look at the $\jth$ convergents of your expression and 
make a conjecture about the even and odd numbered convergents (you do not need to prove it).
\item (Optional but could be helpful for Problem 3) Write a program in any language to compute 
a continued fraction of an input number up to a given accuracy.
\end{enumerate}

\paragraph*{3. Factoring 21}
Let's factor the number 21 using Shor's algorithm.
\begin{enumerate}
  \item List all numbers in $\Z_{21}$ that are relatively prime to $21$.  These are the elements of the multiplicative group $\Z_{21}^\times$.  Compute 
  the order $\ord_{21}(x)$ of all elements in $\Z_{21}^\times$.
  \item Choose an $x \in \Z_{21}^\times$ with $\ord_{21}(x) = 6$.  See that $\gcd(x\pm 1, n)$ gives a nontrivial factor of $n$.
  \item Now let's simulate finding the order of $f(j) = x^j \bmod{21}$ for the $x$ you chose in the last step.  Using the Octave FTperiod program 
  \footnote{Currently I have only added the sampling functionality to the Matlab/Octave program.  If I have time I will also add it to the python version.  
  Octave programs can be run online at \url{https://octave-online.net/}.}
  \url{https://github.com/troyjlee/qalgo/tree/main/CODE} with $N = 21^2, s=6$.  This simulates randomly sampling a state $\ket{g_t}$ and 
  measuring $F_N \ket{g_t}$ to see an index $b$.  Use continued fraction expansion on $b/N$ and see if you can recover $\ord_{21}(x)$.  
  It may take several attempts.  Record the values you see and how many attempts it takes.
\end{enumerate}

\paragraph*{4. Assumptions} 
Where in the proof of correctness of Shor's algorithm for the general period finding problem with a function 
$f: \Z_N \rightarrow [M]$ do we use the assumption that $N > M^2/2$?  What can go wrong without this assumption?

\paragraph*{5. Cosets}
Let $G$ be a finite group and $K, L \le G$ subgroups of $G$.  For $a,b \in G$ let $aK = \{a \cdot k : k \in K\}$ 
be a left coset of $K$ and $bL$ similarly be a left coset of $L$.  If $d = |K \cap L|$ show that $|aK \cap bL| \in \{0,d\}$.  

\paragraph*{6. Finding all ones}
Let $N = 2^n$ and $x \in \{0,1\}^N$ and \emph{assume you know} that $x$ has $k$ many ones.  
\begin{enumerate}
\item In lecture we showed how to find an $i \in N$ such that $x_i =1$ with constant probability by a quantum algorithm after $O(\sqrt{N/k})$ many queries to $x$.  
Show how to boost this success probability to $1-1/N^2$ using $O(\sqrt{N/k} \log(N))$ many queries to $x$.
\item Give a quantum algorithm  to find \emph{all} the ones in $x$ with constant probability after $O(\sqrt{kN} \log(N))$ many queries to $x$.
\end{enumerate}

\paragraph*{7. Exact searching}
Do Exercise~4 in Chapter~7 of Ronald de Wolf's lecture notes \url{https://arxiv.org/abs/1907.09415}.  For part~(c) you may assume 
you have access to the phase oracle $O_{f,\pm}$ for $f$ and may use extra ancillas and any elementary gates you like.
\end{document}
















