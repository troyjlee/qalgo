\documentclass[12pt]{article}

\usepackage{times}
\usepackage{fullpage}
\usepackage{amsfonts,amssymb,amsmath,amsthm}
\usepackage{latexsym}
\usepackage{gauss}
\usepackage{tikz}
\usepackage{tikz-cd}
\usepackage{tkz-graph}
\usepackage{calc}
\usepackage{url}
\usepackage{thmtools}
\usepackage{thm-restate}
\usepackage{hyperref}
\usepackage[capitalise, nameinlink]{cleveref}
\usepackage{algorithm}
\usepackage{algorithmicx}
\usepackage{algpseudocode}
%\usepackage[pdftex]{color}
%\usepackage[breaklinks]{hyperref}
\usepackage{graphicx}
\usepackage{enumitem}
\setcounter{MaxMatrixCols}{15}

%%% Basic notation
\newcommand{\N}{\mathbb{N}}
\newcommand{\Q}{\mathbb{Q}}
\newcommand{\R}{\mathbb{R}}
\newcommand{\Z}{\mathbb{Z}}
\newcommand{\F}{\mathbb{F}}
\newcommand{\E}{\mathbb{E}}
\newcommand{\C}{\mathbb{C}}
\newcommand{\zero}{\mathbf{0}}
\newcommand{\Id}{\mathbf{I}}
\newcommand{\vecu}{\vec{u}}
\newcommand{\Acal}{\mathcal{A}}
\newcommand{\Ccal}{\mathcal{C}}
\newcommand{\Hcal}{\mathcal{H}}
\newcommand{\Lcal}{\mathcal{L}}
\newcommand{\Pcal}{\mathcal{P}}
\newcommand{\Rcal}{\mathcal{R}}
\newcommand{\Kcal}{\mathcal{K}}
\newcommand{\M}{\mathcal{M}}
\newcommand{\Scal}{\mathcal{S}}
\newcommand{\Tcal}{\mathcal{T}}
\newcommand{\Fcal}{\mathcal{F}}
\newcommand{\Gcal}{\mathcal{G}}
\newcommand{\Tr}{\mathrm{Tr}}
\newcommand{\bin}{\mathrm{bin}}
\newcommand{\iu}{\mathrm{i}}
\newcommand{\OR}{\mathrm{i}}
\newcommand{\MAJ}{\mathrm{MAJ}}
\newcommand{\symvec}{\mathrm{symvec}}
\newcommand{\symmat}{\mathrm{Sym}}
\newcommand{\Recover}{\mathsf{RecoverOneFromAll}}
\newcommand{\NULL}{\mathrm{NULL}}
\newcommand{\MINCUT}{\mathrm{MINCUT}}
\newcommand{\CON}{\mathrm{CONNECTIVITY}}
\newcommand{\THRESH}{\mathrm{THRESHOLD}}
\newcommand{\Contract}{\mathrm{Contract}}
\newcommand{\ARGMINCUT}{\mathrm{ARGMINCUT}}
\newcommand{\lin}{\mathrm{lin}}
\newcommand{\cut}{\mathrm{cut}}
\newcommand{\ind}{\mathrm{ind}}
\newcommand{\diag}{\mathrm{diag}}
\newcommand{\IP}{\mathrm{IP}}
\newcommand{\atoms}{\mathrm{atoms}}
\newcommand{\PARITY}{\mathrm{PARITY}}
\newcommand{\upto}{\mathbin{:}}
\DeclareMathOperator*{\argmin}{argmin}
\newcommand{\polylog}{\mathrm{polylog}}
\newcommand{\zeros}{\mathrm{zeros}}
\newcommand{\ones}{\mathrm{ones}}
\newcommand{\rmvec}{\mathrm{vec}}
\newcommand{\rmmat}{\mathrm{mat}}
\newcommand{\laspan}{\mathrm{span}}
\newcommand{\cross}{\mathrm{overlap}}
\newcommand{\coisa}{weighted biadjacency matrix }
\newcommand{\card}[1]{|#1|}
\newcommand{\len}[1]{l(#1)}
\newcommand{\floor}[1]{\lfloor #1 \rfloor}
\newcommand{\ceil}[1]{\left\lceil #1 \right\rceil}
\newcommand{\pair}[2]{\langle #1,#2 \rangle}
\newcommand{\triple}[3]{\langle #1,#2,#3 \rangle}
\def\01{\{0,1\}}

\newcommand{\ith}{i^{\scriptsize \mbox{{\rm th}}}}
\newcommand{\jth}{j^{\scriptsize \mbox{{\rm th}}}}

%\newcommand{\dim}{\mathrm{dim}}
\newcommand{\cd}{\mathrm{cdim}}
\newcommand{\sep}{\mathrm{sep}}
\newcommand{\mer}{\mathrm{mer}}

\DeclareMathOperator*{\argmax}{arg\,max}

\newcommand{\norm}[1]{\|#1\|}


\newcommand{\troy}[1]{\textcolor{red}{#1}}

\newcommand{\bra}[1]{\langle#1|}
\newcommand{\ket}[1]{|#1\rangle}
\newcommand{\bigket}[1]{\big|#1\big\rangle}
\newcommand{\braket}[2]{\langle#1, #2\rangle}
\newcommand{\rk}{\mathrm{rk}}
\newcommand{\I}{\mathcal{I}}
\newcommand{\1}{\mathbf{1}}


\newtheorem{theorem}{Theorem}
\newtheorem{question}[theorem]{Question}
\newtheorem{lemma}[theorem]{Lemma}
\newtheorem{corollary}[theorem]{Corollary}
\newtheorem{proposition}[theorem]{Proposition}
\newtheorem{remark}[theorem]{Remark}
\newtheorem{example}[theorem]{Example}
\newtheorem{claim}[theorem]{Claim}
\newtheorem{fact}[theorem]{Fact}
\newtheorem{hypothesis}[theorem]{Hypothesis}

\theoremstyle{definition}
\newtheorem{exercise}[theorem]{Exercise}
\newtheorem{problem}[theorem]{Problem}
\newtheorem{definition}[theorem]{Definition}

\begin{document}
\title{Problem Set 3}
\date{}
\maketitle

\paragraph*{1. Parity}
$\PARITY_n : \{0,1\}^n \rightarrow \{0,1\}$ is the function where $\PARITY_n(x) =1$ iff the 
number of ones in $x$ is odd.  
\begin{enumerate}
  \item Show that $\PARITY_2$ can be solved exactly with one quantum query.  Hint: This is just Deutsch-Josza.
  \item Show that $\PARITY_n$ can be solved exactly with $\ceil{n/2}$ many quantum queries.  No need to write out 
  circuits here, keep your description of the algorithm high level.
  \item Use the polynomial method to prove that $Q_{1/3}(\PARITY_n) \ge \ceil{n/2}$.  
\end{enumerate}

\paragraph*{2. Dual polynomials}
In lecture we only saw techniques to lower bound the approximate degree of \emph{symmetric} functions.  Proving 
lower bounds on the approximate degree of functions which aren't symmetric is challenging.  One way to do this is by \emph{dual polynomials}, 
which introduce here.  

Let $f: \{-1,1\} \rightarrow \{0,1\}$.  
Show that if there exists a function $g: \{-1,1\}^n \rightarrow \R$ with the properties
\begin{enumerate}
  \item $\sum_{x \in \{-1,1\}^n} g(x) f(x) > \frac{1}{3} \sum_{x \in \{-1,+1\}^n} |g(x)|$
  \item $\sum_{x \in \{-1,1\}^n} g(x) \chi_S(x) = 0$ for all $S \subseteq \{1, \ldots, n\}$ with $|S| \le d$
\end{enumerate}
then $\deg_{1/3}(f) > d$.  

In the dual polynomial method, one explicitly constructs a function $g$ satisfying these properties for as 
large a $d$ as possible. 

\paragraph*{3. Simple version of the adversary method}
The Hamming distance $d_H(x,y)$ between two strings $x,y \in \{0,1\}^n$ is the number of positions on which they differ, 
that is $d_H(x,y) = |x \oplus y|$.  

Let $f : \{0,1\}^n \rightarrow \{0,1\}$.  Suppose that for every $x \in f^{-1}(0)$ there are at least $d_0$ many $y \in f^{-1}(1)$ 
with $d_H(x,y) = 1$ and that for every $y \in f^{-1}(1)$ there are at least $d_1$ many $x \in f^{-1}(0)$ with $d_H(x,y) = 1$.  
Show that the quantum adversary bound for $f$ is at least $\sqrt{d_0 d_1}$.  In other words, construct a $|f^{-1}(0)|$-by-$|f^{-1}(1)|$ 
matrix $\Gamma$ with 
\[
\frac{\|\Gamma\|}{\max_{i \in \{1, \ldots, n\} } \|\Gamma \circ D_i\|} \ge \sqrt{d_0 d_1} \enspace .
\]
Hint: You can take all entries of $\Gamma$ to be in $\{0,1\}$.  Useful characterizations of the spectral norm of a matrix $A \in \R^{m \times n}$ include
\begin{enumerate}
  \item $\| A \| = \max_{v \in \R^n \atop \|v\| = 1} \|Av \|$
  \item $\| A\| = \max_{u \in \R^m, v \in \R^n \atop \|u\| = \|v\| = 1} |u^T A v|$
  \item $\| A \| = \sqrt{ \lambda_1(AA^T)}$ where $\lambda_1(B)$ is the largest eigenvalue of $B$.
\end{enumerate}

\paragraph*{4. Applying the simple adversary bound}
\begin{enumerate}
  \item Use the simple version of the adversary method to show that $Q_{1/3}(\PARITY_n) = \Omega(n)$.
  \item For $n$ a positive integer and $1 \le k \le n$ let $\THRESH_{k,n}$ be a \emph{partial} Boolean function with domain $\{x \in \{0,1\}^n : |x| \in \{k-1, k\}\}$ 
  and where $\THRESH_{k,n}(x) = 1$ iff $|x| = k$.  Use the simple version of the adversary method to show that $Q_{1/3}(f_{n,k}) = \Omega(\sqrt{k(n-k)})$.  
  Give a quantum query algorithm to show that this lower bound is tight up to logarithmic factors (hint: use one of the algorithms from the last problem set).  
\end{enumerate}

\paragraph*{5. Not All Equal}
Let $f: \{-1,1\}^3 \rightarrow \{-1,+1\}$ be the Not-All-Equal function, which evaluates to $-1$ on 
input $x \in \{-1,1\}^3$ if not all the entries of $x$ are equal and evaluates to $1$ otherwise.  In other 
words, it evaluates to $1$ on the two inputs $111, -1 -1 -1$, and evaluates to $-1$ otherwise.  
\begin{enumerate}
  \item Write $f$ as a polynomial.  What is its degree?
  \item Show that any $1/3$-error approximating polynomial for $f$ has degree at least $2$.  
  \item Give a 2 query quantum algorithm that computes $f$ with success probability $1$.
  \item Challenge: Show that there is no quantum algorithm that computes $f$ with success probability $1$ 
  using just 1 query.
\end{enumerate}

\paragraph*{6. Element Distinctness}
Let $n$ be a positive integer and $M \ge n$.  Element distinctness $\ED_n : \{0, \ldots, M-1\}^n \rightarrow \{0,1\}$ is the function where 
$\ED(x) = 1$ if $x_i \ne x_j$ for all $i,j \in \{1, \ldots, n\}$ with $i \ne j$ and $\ED(x) = 0$ otherwise.  In other words, $\ED(x)=1$ if all the 
elements of $x$ are distinct.
\begin{enumerate}
  \item What is the success probability of the following algorithm: Form a set $S$ by choosing $k$ elements from $\{1, \ldots, n\}$ uniformly at random (with 
  replacement) and then use Grover to search for $j \not \in S$ such that $x_j = x_i$ for some $i \in S$?
  \item Use part~1 and amplitude amplification to show $Q_{1/3}(\ED_n) = O(n^{3/4})$.  See Section~1.1 of the lecture notes on Grover's algorithm for a description of 
  amplitude amplification.
\end{enumerate}

\end{document}
















